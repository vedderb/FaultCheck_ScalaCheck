\section{The FaultCheck distribution}\label{the-faultcheck-distribution}

This collection of software consists of the tool FaultCheck, which
provides fault injection support for property-based testing tools, and
one use case. In the use case, a generic simple end-to-end library
written in C is tested with ScalaCheck. There are also several
property-based tests to evaluate whether FaultCheck fulfils its
specification based on ScalaCheck.

\section{Scripts}\label{scripts}

This distribution contains several scripts to automate certain tasks.
Their purpose is as follows:

\subsection{Directory: /}\label{directory}

\textbf{set\_env}\\Set up the environment for running FaultCheck and the
E2E example where the dynamic libraries are looked up and the includes
are resolved.

\textbf{build}\\Compiles FaultCheck and the C files for the example(s).

\textbf{build\_doc}\\Generates the documentation from the README.md
markup file. The output appears in the \emph{Documentation} directory.

\textbf{clean}\\Removes the object files from the build process.

\textbf{clean\_scala}\\Removes all the bytecode files generated by the
ScalaCheck example and the ScalaCheck tests.

\subsection{Directory:
/Examples/E2E\_Scala}\label{directory-examplese2eux5fscala}

\textbf{gen\_coverage}\\Generate C code coverage reports in HTML.

\textbf{clean\_coverage}\\Remove the coverage files.

\section{Installing dependencies}\label{installing-dependencies}

In order to run this software, make sure that all libraries that are
required are installed. For FaultCheck itself, the Qt SDK is required
and for the examples the GNU C compiler is required. For the scala
example and the FaultCheck tests, the sbt tool is required. To generate
the coverage reports and the documentation, some additional tools are
required.

\subsection{Ubuntu Linux}\label{ubuntu-linux}

Setting everything up on a fresh Ubuntu install is fairly simple. First,
install some dependencies:

\begin{quote}
sudo apt-get install build-essential qt-sdk qt4-qmake
\end{quote}

\textbf{ScalaCheck examples and tests}\\For the ScalaCheck examples and
tests, install scala, openjdk and sbt:

\begin{quote}
wget http://apt.typesafe.com/repo-deb-build-0002.deb\\sudo dpkg -i
repo-deb-build-0002.deb\\sudo apt-get update\\sudo apt-get install sbt
scala openjdk-7-jdk
\end{quote}

\textbf{Coverage information}\\To generate coverage information with the
provided script for the examples, install lcov 1.10 or later. On Ubuntu
14.04, the version from the repositories can be used:

\begin{quote}
sudo apt-get install lcov
\end{quote}

For earlier versions of Ubuntu, download and install an updated version
of lcov from
here:\\\url{http://packages.ubuntu.com/trusty/all/lcov/download}

\textbf{Documentation}\\To generate LaTeX, PDF and HTML documentation
from the markdown file, install pandoc and some LaTeX tools:

\begin{quote}
sudo apt-get install pandoc texlive-latex-base cm-super
\end{quote}

\section{Setting up the environment}\label{setting-up-the-environment}

From the root directory, set up the environment with the provided scrip.
The script has to be run with source so that it is executed from the
current shell:

\begin{quote}
source set\_env
\end{quote}

Build everything:

\begin{quote}
./build
\end{quote}

At this point, the ScalaCheck example and the functionality tests should
work from the current shell. Note that the environment has to be set
with the source command every time a new shell is started. Also note
that the CopterSim scripts set up the environment themselves, so they
can be run independently.

\section{E2E example with ScalaCheck}\label{e2e-example-with-scalacheck}

There is one example where fault injection on a generic end-to-end
library written in C is performed. The test is written in Scala and uses
ScalaCheck.

Go to \emph{Examples/E2E\_Scala/scala} and run the test

\begin{quote}
sbt run
\end{quote}

The first time this is done, the compilation and dependency fetching of
sbt might take some time. After the compilation a lot of output is
printed from the test. In the end, the test should pass.

To switch the E2E-library on and off, open
\emph{Examples/E2E\_Scala/scala/src/main/scala/e2e/AirbagTest.scala} and
change the variable near the beginning:

\begin{quote}
val useE2E
\end{quote}

Switching the E2E-library off and running the test again should reveal
some failing test cases.

\subsection{Generate code coverage
information}\label{generate-code-coverage-information}

The \emph{gen\_coverage} script in the corresponding example can be used
to generate coverage information in the \emph{Coverage} directory. This
can only be done after the experiment has been run. In
\emph{Examples/E2E\_Scala/scala/src/main/scala/e2e/AirbagTest.scala}
different faults can be switched on and off (see the comments) to
exercise different or all fault handling mechanisms.

\section{Testing FaultCheck}\label{testing-faultcheck}

There are some property-based tests that can be used to verify that
FaultCheck works as intended. These tests are implemented in ScalaCheck.
When new features are added to FaultCheck, these tests should be updated
and/or re-run. Currently, there are two different tests: one for the
probing-part of FaultCheck and one for the packet-based communication
channel of FaultCheck. For these tests to work, everything has to be
built and the environment has to be set up as described above.

\subsection{Testing the probing part}\label{testing-the-probing-part}

Go to \emph{Tests/General} and run the test with

\begin{quote}
sbt run
\end{quote}

All the tests should pass.

\subsection{Testing the communication
channel}\label{testing-the-communication-channel}

Go to \emph{Tests/Packet} and run the test with

\begin{quote}
sbt run
\end{quote}

All the tests should pass.

\section{Generating documentation}\label{generating-documentation}

Currently, this tutorial is written with the markdown syntax in the
\emph{README.md} file and the files in the Documentation directory are
generated with the \emph{build\_doc} script. To update the
documentation, edit the \emph{README.md} file and run the script:

\begin{quote}
./build\_doc
\end{quote}

Make sure that the dependencies for this script are installed as
described above.
